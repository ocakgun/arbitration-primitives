\documentclass[conference]{IEEEtran}
\usepackage[pdftex]{graphicx}
\graphicspath{{../pdf/}{../jpeg/}}
\DeclareGraphicsExtensions{.pdf,.jpeg,.png}
\usepackage{url}

% correct bad hyphenation here
\hyphenation{op-tical net-works semi-conduc-tor}

\begin{document}
\title{Asynchronous Arbitration Primitives\\for New Generation of Circuits and Systems}

\author{\IEEEauthorblockN{Andrey Mokhov\IEEEauthorrefmark{1}, Victor Khomenko, Danil Sokolov, Alex Yakovlev}
\IEEEauthorblockA{Newcastle University, Newcastle upon Tyne, United Kingdom}
\IEEEauthorblockA{\IEEEauthorrefmark{1}\emph{Corresponding author:} \url{andrey.mokhov@ncl.ac.uk}}}

\maketitle

\begin{abstract}
This paper presents an overview of a family of asynchronous arbitration
primitives designed to increase the resilience and efficiency of
the new generation of circuits and systems. We cover primitives for
interfacing analog and digital worlds, sampling of non-persistent
signals, and efficient handling of correlated sensor events.
\end{abstract}

% no keywords

\section{Introduction}

TODO


\section{Synchronisation primitives}

In this section we cover \emph{synchronisation primitives} that are used to
isolate asynchronous control logic from potentially hazardous environment.

\subsection{WAIT and WAIT0}

\subsection{RWAIT and RWAIT0}

\subsection{WAIT01 and WAIT10}

\subsection{WAIT2}

% \begin{figure*}[ht!]
% \begin{center}
%     \includegraphics[width=0.96\linewidth]{FIG/screen.png}
%     \vspace{-3.5mm}
%     \caption{From formal specification to hardware synthesis and
%     simulation of a simple 3-instruction processing core. A screenshot of \textsc{Workcraft}.}
%     \label{fig:screenshot}
% \end{center}
% \vspace{-7.5mm}
% \end{figure*}

\section{Decision-making primitives}

This section presents a family of \emph{decision-making} components that perform
non-trivial event arbitration and coordination tasks and rely on the previously
introduced synchronisation primitives.

\subsection{WAITX}
\subsection{SAMPLE}
\subsection{OM}

% \begin{figure}[h!]
% \begin{center}
%   \includegraphics[width=0.84\linewidth]{FIG/ope-chip.pdf}
%   \vspace{-3mm}
%   \caption{Resiliency of asynchronous control under unstable voltage.}
%   \label{fig:voltage-resiliency}
% \end{center}
% \vspace{-7mm}
% \end{figure}

\section{Conclusions}

\section*{Acknowledgements}

\bibliography{publications}

\end{document}
